% !TeX spellcheck = de_DE
% !TEX encoding = UTF-8 Unicode
%%%%%%%%%%%%%%%%%%%%%%%%%%%%%%%%%%%%%%%%%%%%%%%%%%%%
% The first part of the header needs to be copied
%       into the note options in Anki.
%%%%%%%%%%%%%%%%%%%%%%%%%%%%%%%%%%%%%%%%%%%%%%%%%%%%

% layout in Anki:
\documentclass[9pt]{article}
\usepackage[a4paper]{geometry}
\geometry{paperwidth=.5\paperwidth,paperheight=100in,left=2em,right=2em,bottom=1em,top=2em}
\pagestyle{empty}
\setlength{\parindent}{0in}
 
% hyphenation:
\usepackage[ngerman]{babel}

% encoding:
\usepackage[T1]{fontenc}
\usepackage[utf8]{inputenc}
\usepackage{lmodern}

% packages:
\usepackage{parskip}
\usepackage[free-standing-units=true]{siunitx}
% Writng si units, numbers, list of numbers etc.
\usepackage{gensymb}
% Unified typseting of units outside of siuitx
\usepackage{amsmath}
\usepackage{amsfonts}  
% Math features
\usepackage{esdiff}
% Typeseting of (partial)derivatives 
\usepackage{commath}
% More  derivatives. Not so nice like esdiif but adds
% \dif comand for upright d in math mode.
\usepackage{bm}
% Bold font in math mode
\usepackage{esint}
% Some fancy integrals signs. Mutilpe integrals
\usepackage{enumerate}
% Different styles for enumerate lists
\usepackage{multirow}
% More advanced tabular
\usepackage{physics}
% Lots of usefull comannds for physicists. Vektors, nablas etc.
\usepackage{amssymb, xfrac,bbold}
% extra fonts and symbols 
\usepackage{mathtools}
% extension to amsmath, fixes, meany new tool
\usepackage{isotope}
\usepackage{empheq}
%commands:
\usepackage{ifthen}

\newcommand*{\tran}{^{\mkern-1.5mu\mathsf{T}}}
\newcommand{\Rn}{\mathbb{R}^n}
\newcommand{\Rk}{\mathbb{R}^k}
\newcommand{\defeq}{\coloneqq}
\newcommand{\R}[1]{%
\ifthenelse{\equal{#1}{}}%
	{\mathbb{R}}
	{\mathbb{R}^{#1}}}
\newcommand{\C}[1]{%
	\ifthenelse{\equal{#1}{}}
	{\mathbb{C}}
	{\mathbb{C}^{#1}}}%	
\renewcommand{\vec}[1]{\underline{#1}}
\DeclarePairedDelimiter{\innerprod}\langle\rangle
%\DeclarePairedDelimiter{\norm}{\lVert}{\rVert} 
\newcommand{\Fr}{\mathcal{F}}
\newcommand{\Hi}{\mathcal{H}}
%%%%%%%%%%%%%%%%%%%%%%%%%%%%%%%%%%%%%%%%%%%%%%%%%%%%
% Following part of header NOT to be copied into
%            the note options in Anki.
%          ! Anki will throw an errow !
%%%%%%%%%%%%%%%%%%%%%%%%%%%%%%%%%%%%%%%%%%%%%%%%%%%%%
%
%  pdf layout:
%
\geometry{paperheight=74.25mm}
\usepackage{pgfpages}
\pagestyle{empty}
\pgfpagesuselayout{8 on 1}[a4paper,border shrink=0cm]
\makeatletter
\@tempcnta=1\relax
\loop\ifnum\@tempcnta<9\relax
\pgf@pset{\the\@tempcnta}{bordercode}{\pgfusepath{stroke}}
\advance\@tempcnta by 1\relax
\repeat
\makeatother
% 
%  notes, fields, tags:
%
\newcommand{\xfield}[1]{
        #1\par
        \vfill
        {\tiny\texttt{\parbox[t]{\textwidth}{\localtag \ \globaltag\hfill\uuid}}}
        \newpage}
\newenvironment{field}{}{\vfill
        {\tiny\texttt{\parbox[t]{\textwidth}{\localtag \ \globaltag\hfill\uuid}}} \newpage}
\newif\ifnote
\newenvironment{note}{\notetrue}{\notefalse}
\newcommand{\localtag}{}
\newcommand{\globaltag}{}
\newcommand{\uuid}{}
\newcommand{\tags}[1]{
    \ifnote 
        \renewcommand{\localtag}{#1}
    \else
        \renewcommand{\globaltag}{#1}
    \fi 
    }
\newcommand{\xplain}[1]{\renewcommand{\uuid}{#1}}
%
%%%%%%%%%%%%%%%%%%%%%%%%%%%%%%%%%%%%%%%%%%%%%%%%%%%%
% The following line again needs to be copied 
% into Anki:
\begin{document}
%%%%%%%%%%%%%%%%%%%%%%%%%%%%%%%%%%%%%%%%%%%%%%%%%%%%

\tags{KOMOT::Optimierungsprobleme}

\begin{note}
  \xplain{926a7d3e-7e31-45e9-bf49-bd34967981a1}
  % 1.1.2 
  \tags{aufgabenstellung}
  \xfield{Was versteht man unter einer \textit{Zielfunktion}?}
  \begin{field}   
  	Die Funtktion $f: G \rightarrow \R{}$, die minimiert wird. 
  \end{field}
  \xfield{Was ist der \textit{zul"assiger Bereich} $G$ ?}
  \begin{field}
  	Definitionsbereich der Zielfunktion. $G \subseteq \Rn$ .
  \end{field}
  \xfield{Was verstehen wir unter einer (globalen) L"osung einer OA Aufgabe.}
  \xfield{Ein $x^* \in G$ das die Zielfunktion minimiert. } 
  \xfield{Was ist eine \textit{lokale L"osung} $x^*$ einer OA Aufgabe?}
  \begin{field}
  	\begin{equation*}
  		f(x^*) \leq f(x) \quad  \forall x \in G \cap U(x^*) ,
  	\end{equation*}
  	 und es existiert so eine Umgebung von $x^*$.
  \end{field}
  \xfield{Was ist eine \textit{isolierte L"osung}?}
  \begin{field}
  	Es existiert eine Umgebung $U(x*)$, so dass $f(x^*) < f(x)$.
  	Bzw. es gibt keine witeren lokalen Lo"sungen in der Umgebung. 
  \end{field}
  \xfield{ Wie hei"st $f_\text{min} \defeq f(x^*)$ ?}
  \xfield{\textit{Optimalwert} oder \textit{Minimalwert}}
\end{note}

\begin{note}
	\xplain{b453bc66-dbc9-4445-898e-d69f7898f0dd}
	% 1.1.3
	\tags{konvexitaet}
	\xfield{Eine Menge $G\subseteq\Rn$ hei"st \textit{konvex}, wenn ...}
	\begin{field}
		$\forall_{x,y \in G}$ die Verebindungstrecke zwischen den Punkten auch in $G$ liegt. Formel: 
		\[\lambda x + (1 - \lambda )y \in G, \quad \forall (x,y,\lambda ) \in (G \times G \times (0, 1)) \]
	\end{field}
	\xfield{Sei $G$ \textit{konvex}. Eine Funktion $f:G \rightarrow \R{}$ hei"st \textit{konvex auf} $G$, wenn... }
	\begin{field}
	$\forall (x,y,\lambda ) \in (G \times G \times (0, 1))$:
		\[f(\lambda x + (1-\lambda) y) \leq \lambda f(x) + (1 - \lambda) f(y).\]
	\end{field}
	\xfield{Wann ist eine Funktion \textit{streng konvex} auf einer kompakten Menge $G$? }
	\xfield{Wie bei normalen konvexit"at, aber mit $<$ statt $\leq$.}
		\xfield{Sei $G$ \textit{konvex}. Eine Funktion $f:G \rightarrow \R{}$ hei"st \textit{gleichm"a"sig konvex auf} $G$, wenn... }
	\begin{field}
	$\exists \gamma > 0$, so dass $\forall (x,y,\lambda ) \in (G \times G \times (0, 1))$:
		\[f(\lambda x + (1-\lambda) y) \leq \lambda f(x) + (1 - \lambda) f(y) - \gamma \lambda (1 - \lambda ) \norm{x-y}^2 \]
	\end{field}
	\end{note}
	
\begin{note}
	\xplain{4632476d-40e2-47b5-bbca-35f7cf3d5487}
	% 1.1 Lemma
	\begin{field}
		$B$ offen $G$ konvex und $G\subseteq B \subseteq \Rn$, $f: B \rightarrow \R{}$ diff'bar.
		$f$ \textit{konvex} g.d.w. ... 
	\end{field}
	\begin{field}
	    $\forall_{x,y \in G}$: 
		\[f(y) - f(x) \geq \nabla f(x)\tran (y-x). \]
	\end{field}
	\begin{field}
		$B$ offen $G$ konvex und $G\subseteq B \subseteq \Rn$, $f: B \rightarrow \R{}$ diff'bar.
		$f$ \textit{streng konvex} g.d.w. ... 
	\end{field}
	\begin{field}
	    $\forall_{x,y \in G}, x \neq y$: 
		\[f(y) - f(x) > \nabla f(x)\tran (y-x). \]
	\end{field}
	\begin{field}
		$B$ offen $G$ konvex und $G\subseteq B \subseteq \Rn$, $f: B \rightarrow \R{}$ diff'bar.
		$f$ \textit{gleichm"a"sig konvex} g.d.w. ... 
	\end{field}
	\begin{field}
	    $\exists \gamma >0$, so dass $\forall_{x,y\in G}$: 
		\[f(y) - f(x) \geq \nabla f(x)\tran (y-x) + \gamma\norm{x - y}^2. \]
	\end{field}
\end{note}

\begin{note}
	\xplain{40b22ce9-aa36-4767-9356-920e9b42b0fb}
	% 1.2
	\tags{definitheit}
	\begin{field}
		Wann ist eine quadratische Matrix  $M \in \R{n \times n }$\textit{positiv semidefinit}?
	\end{field}
	\begin{field}
		wenn $s\tran M s \geq 0 \forall s \in \Rn$. ($\Leftrightarrow$: Alle Eigenwerte $ \geq 0$.)
	\end{field}
	\xfield{Wann ist eine quadratische Matrix  $M \in \R{n \times n }$\textit{positiv definit}?}
	\begin{field}
		wenn $s\tran M s > 0, \quad \forall s \in \Rn / \set{0} $. 
		($\Leftrightarrow$: Alle Eigenwerte $ > 0$.)
	\end{field}
\end{note}

\begin{note}
	\xplain{00a8ae21-6359-426c-8779-150f4a27353a}
	\tags{}
	% Lemma 1.2
	\begin{field}
		$G \subseteq \Rn$ offen und konvex, $f: G \rightarrow \R{}$ zweimal stetig diff'bar.
		 Dann $f$ is \textit{konvex} auf $G$, genau dann wenn... 
	\end{field}
	\begin{field}
		$\forall x \in G: \nabla^2f(x)$ positiv semidefinit.
	\end{field}
	\begin{field}
		$G \subseteq \Rn$ offen und konvex, $f: G \rightarrow \R{}$ zweimal stetig diff'bar.
		 Dann $f$ is \textit{streng konvex} auf $G$, genau dann wenn... 
	\end{field}
	\begin{field}
		$\forall x \in G: \nabla^2f(x)$ positiv definit.
	\end{field}
	\begin{field}
		$G \subseteq \Rn$ offen und konvex, $f: G \rightarrow \R{}$ zweimal stetig diff'bar.
		 Dann $f$ is \textit{gleichm"a"sig konvex} auf $G$, genau dann wenn... 
	\end{field}
	\begin{field}
		$\exists \gamma > 0$, so dass $\forall s,x \in G$:
		\[s\tran \nabla^2f(x)s \geq \gamma \norm{s}^2\] 
	\end{field}
\end{note}

\begin{note}
	\xplain{3549ab37-ceda-4214-a3af-35470bbd7690}
	\tags{}
	%Theorem 1.1
	\begin{field}
		Die Zielfunktion sei konvex, was gibt uns das( bezogen auf L"osugnen)?
	\end{field}
	\begin{field}
    		Jede lokale L"osung ist auch eine globale L"osung
	\end{field}
	\begin{field}
		Die Zielfunktion sei streng konvex, was gibt uns das( bezogen auf L"osugnen)?
	\end{field}
	\begin{field}
		Es gibt h"ochstens eine globale L"osung.
	\end{field}
	\begin{field}
		Die Zielfunktion sei gleichm"a"sig konvex, was gibt uns das ( bezogen auf L"osugnen)?
	\end{field}
	\begin{field}
		Falls $G$ nicht nur konvex aber auch abgeschlossen und nichtleer, dann besitzt die OA \emph{genau eine} L"osung. 
	\end{field}
\end{note}

\begin{note}
	\xplain{4d62a7fe-78a0-4188-be46-aea0d45ea53c}
	\tags{}
	\begin{field}
		Was ist die Definition der \emph{quasikonvexit"at}?
	\end{field}
	\begin{field}
		$G \subseteq \Rn$ konvex. $f:G \rightarrow \R{}$ hei"st \textit{quasikonvex} auf $G$, wenn 
		\[f( \lambda x + (1- \lambda ) y) \leq \max\set{f(x), f(y) }\]
	\end{field}
	\begin{field}
	Was bedeutet, dass eine Funktion \textit{pseudokonvex} ist?
	\end{field}
	\begin{field}
		Sei $G$ konvex, $B$ offen mit $G \subseteq B \subseteq \Rn$. Sei $f: B \rightarrow \R{}$ diff'bar.
		$f$ ist pseudokonvex auf G, wenn $\forall x,y \in G$:
		\[(y-x)\tran\nabla f(x) \geq 0 \Rightarrow f(y) \geq f(x).\]
	\end{field}
	\xfield{Was ist st"arker, \textit{pseudokonvexit"at} oder \textit{quasikonvexit"at}?}
	\xfield{\textit{pseudokonvexit"at}}
\end{note}

%1.2
\tags{KOMOT::Optimalitaetsbedinugngen}
%1.2.1
\begin{note}
	\xplain{d402bf82-614a-4943-9ae2-d633802088eb}
	\tags{}
	\xfield{Definiere den Kegel der zul"assigen Richtungen in $x\in G$.}
	\begin{field}
		\[Z(x) \defeq \text{cone}\set{d \in \Rn \ | \ x + \alpha d \in G, \ \ 
		\forall \alpha \in \left[0, 1 \right] } \]
		, wobei $\text{cone}(S) \defeq \set{\lambda s \ | \ s\in S, \ \ \lambda \in\left[0, \infty \right)}$.
	\end{field}
	\begin{field}
		Sei $G$ konvex, die Zielfunktion $f$ [... (1)], $x^*$ [... (2)], und es gilt [... (3)], dann ist $x^*$ eine globale L"osung der OA.
	\end{field}
	\begin{field}
		\begin{enumerate}
			\item pseudokonvex
			\item  eine lokale L"osung
			\item $\nabla f(x^*)\tran(x-x^*) \geq 0, \ \ \forall x \in G $
		\end{enumerate}
	\end{field}
\end{note}

% 1.2.2
\begin{note}
	\xplain{5a2815a9-22bb-4242-a5fc-bc76956aba84}
	\tags{tangentialkegel}
	\xfield{Definiere den \textit{Tangentialkegel}}
	\begin{field}
		\begin{align*}
		T(x) \defeq \bigg\{ d = \lim_{v\rightarrow \infty} \frac{x^v-x}{t_v} \ \Big | \ 
		\set{x^v} \subset G, \set{t_v} \subset (0, \infty)  , \\
		 \lim_{v \rightarrow \infty} x^v = x, \ \lim_{v \rightarrow \infty} t_v = 0 \bigg\} 				
		\end{align*}
	\end{field}
	\xfield{Wann ist $T(x)$ \textit{Tangentialkegel} gleich $\Rn$?}
	\xfield{Falls $x$ im inneren von $G$ ist.}
\end{note}

%1.2.3
\begin{note}
	\xplain{32005844-5dd8-4870-957c-75e86e72ab82}
	\tags{linearisierungskegel}
	\xfield{Definiere den \textit{Linearisierungskegel}.}
	\begin{field}
		\begin{align*}
			L(x) \defeq \set{ d \ \big| \ \nabla g_i(x)\tran d \leq 0 \text{ f"ur }  \ i \in I_0(x),  \nabla h(x)\tran d =0 }
		\end{align*}
		mit $I_0(x) \defeq \set{i \in I \ | \ g<_i(x) = 0 }$.
	\end{field}
	\xfield{Wie hei"st $I_0$ aus der Definition des \textit{Linearisierungskegels}?}
	\xfield{Indexmenge der in x \textit{aktiven (Ungleichungs)restriktionen}}
\end{note}

%Def. 1.4
\begin{note}
	\xplain{92a7c841-e7f9-4094-b889-3b7adceb6a0f}
	\tags{}
	\xfield{ Wie nennt man die Bedingung, dass $T(x) = L(x)$? }
	\xfield{ \textit{Abadie Constraint Qualification (\textbf{ACG})}}
	\xfield{ Wie nennt man die Bedingung, dass $\bar{\text{conv}\left(T(x)\right)} = L(x)$?}
	\xfield{\textit{Guignard Constraint Qualification (\textbf{GCQ})}}
	\xfield{ Wann ist \textit{ACG} erf"ullt?}
	\xfield{ Falls $T(x) = L(x)$ }
	\xfield{Wann ist \textit{GCQ} erf"ullt?}
	\xfield{ Falls $L(x)$ gleich der abgeschlossener konvexen H"ulle }
\end{note}

\begin{note}
	\xplain{10ae50eb-f330-4302-acf3-37722008f125}
	\tags{}
	\xfield{Wie hei"st eine Bedingung die \textit{ACQ} impliziert?}
	\xfield{Regularit"atsbedingung}
	\xfield{ Nenne f"unf Regularit"atsbedingungen.}
	\begin{field}
		\begin{description}
		\item[EBCQ] Error Bound Constraint Qualification
		\item[MFCQ] Mangasarian--Fromori Constraint Qualification
		\item[LICQ] Linear Independence Constraint Qualification
		\item[Slater Bedinugng]
		\item[Affinit"at] $g_i, i\in I_0(x)$  und $h$ sind affin.
		\end{description}
	\end{field}
	\xfield{Wann gilt \textit{EBCQ}?}
	\begin{field}
		$\exists \delta >0, C > 0$, sodass $\forall z \in B(x, \delta)$
		\[\text{dist}[z, G] \leq C \qty(\norm{\max\set{0, g(z)}} + \norm{h(z)})\]
	\end{field}
	\xfield{Wann gilt \textit{MFCQ}?}
	\begin{field}
		Die Vektoren $\nabla h_1(x), \ldots, \nabla h_p(x)$ sind linear unabh"angig und es gibt ein $s\in \Rn$, so dass
		\[\nabla g_i(x)\tran s > 0, \ \  \forall i \in I_0(x) \]
		und
		\[\nabla h(x)\tran s = 0 . \] 
	\end{field}
	\xfield{Wann gilt \textit{LICQ}?}
	\begin{field}
		Die Vektoren in der Familie 
		\[\set{\nabla g_i(x) \ I \ i\in I_0(x)}  \cup \set{\nabla h_j(x) \ | \ j \in J}\]
		sind linear unabh"angig.
	\end{field}
	\xfield{Wann gilt die \textit{Slater Bedingung}?}
	\begin{field}
		Die Funktionen $g_1, \ldots, g_m$ sind \textbf{konvex} und $J = \emptyset$.
		Au"serdem $\exists \bar{x} \in \Rn$, so dass $g_i(\bar{x}) <0 \ \ \forall i \in I$.
	\end{field}
\end{note}

%1.2.4
\begin{note}
	\xplain{a40dda75-1e4f-4d36-aa19-04733c69739d}
	\tags{KKT}
	\xfield{Wie Lautet die KKT Bedingung (kurz)?}
	\xfield{$\nabla f(x^*)\tran d \geq 0 \ \ \forall d \in L(x^*)$}
	\xfield{ Was besagt \textit{das Lemma von Farkas}?}
	\begin{field}
		Seien $A\in \R{n \times m}, \ B\in\R{n \times p}, \ c\in \Rn$. Dann ist von den folgenden zwei Systemen
		\begin{gather}
			A \tran z \leq 0,\  B \tran z = 0, \ c \tran z > 0 \\
			Au + Bv =c, \ \ u\geq 0
		\end{gather}
		stets genau  ein l"osbar.
	\end{field}
	\xfield{ Wann gielten die KKK-Bedingungen ?}
	\xfield{GCQ muss erf"ullt werden. Man sucht nach einer Regularit"atsbedingung.}
	\xfield{ Schreibe das KKT System auf}
	\begin{field}
	\begin{gather*}
		\nabla_x \mathcal{L}(x, u, v) = 0 \\ 
		h(x) =0 \\ 
		g(x) \leq 0 \\
		u \geq 0 \\
		u \tran g(x) = 0 
	\end{gather*}
		und $u \in \R{m}, v \in \R{p}$ 
	\end{field}
	\xfield{ Wie ist die Lagrangefunktion definiert?}
	\begin{field}
		\[\mathcal{L}(x,u,v) = f(x) + u \tran g(x) + v \tran h(x)\]
		$\forall (x,u,v) \in \Rn \times \R{m} \times \R{p}$
	\end{field}
	\xfield{ Sei $(x^*, u^*, v^*)$ ein KKT--Punkt, wann ist $x^*$ eine globale L"osung?}
	\xfield{ Falls $f$ \textit{pseudokonvex}, alle $g_i$ \textit{quasikonvex} und $h_j$ \textit{affin linear}. }
\end{note}

%1.2.5 
\begin{note}
	\xplain{4648e945-f687-450e-b55e-e255cc8165db}
	\tags{Sattelpunkte}
	\xfield{Definiere einen \textit{Sattelpunkt} einer Lagrangefuntion.}
	\begin{field}
		Ein Punkt $(x^*, u^*, v^*) \in \Rn \times \R{m}_+ \times\R{p}$ hei"st \textit{Sattelpunkt der Langrange Funtion} $\mathcal{L}$, wenn
		\[\mathcal{L}(x^*,u,v) \leq \mathcal{L} (x^*, u^*, v^*) \leq \mathcal{L}(x, u^*, v^*)\]
		$\forall (x,u,v) \in \Rn \times \R{m}_+ \times \R{p}$.
	\end{field}
	\xfield{ Sei $(x^*,u^*, v^*)$ ein Sattelpunkt von $\mathcal{L}$, dann \dots}
	\xfield{\dots $x^*$ eine globale L"osung der OA.}
	
\end{note}

\begin{note}
	\xplain{dd8da85e-d83d-44a0-bb52-e3ea2786ad10}
	\tags{Erweiterte Slater-Bedingung}
	\xfield{Was ist die erweiterte Slater-Bedingung?}
	\begin{field}
		$f: \Rn \rightarrow \R{}, \ g_1, \ldots , g_m: \Rn \rightarrow \R{}$ diff'bar und \textit{konvex}.
		 $h: \Rn \rightarrow \R{p}$ \textit{affin linear}.
	\end{field}
	\xfield{Was gibt die erweiterte Slater-Bedingung ?}
	\xfield{Es gelte die erweiterte Slater-Bedingung. Wenn $x*$ eine L"osung der OA, dann gibt es ein Sattelpunkt mit $x=x^*$.}	
\end{note}

\begin{note}
	\xplain{ef2d2fd9-3a6d-4a9d-9645-646186e59528}
	\tags{Dualtitaet}
	\xfield{Was ist die \textit{primale} Optimierungsaufgabe?}
	\begin{field}
		\[P(x) \rightarrow \min_x\]
		\[P: \Rn \rightarrow \R{} \cup \set{+ \infty} \ \ P(x) \defeq \sup_{u \in \R{m}_+,v \in \R{p}} \mathcal{L}(x, u, v)\]
	\end{field}
	\xfield{Was ist die \textit{duale} Optimierungsaufgabe?}
	\begin{field}
		\[D(x) \rightarrow \max_{u, v}, \ \ \text{bei } u\geq 0\]
		\[D: \R{m} \times \R{p} \rightarrow \R{} \cup \set{+\infty}, \ \  D(u,v) \defeq \inf_{x\in \Rn}\mathcal{L}(x, u, v)\]
	\end{field}
	\xfield{Wie lautet der \textit{Schwacher Dualit"atssatz?}}
	\begin{field}
		\[ \forall (x,u,v) \in \Rn \times \R{m}_+ \times \R{p} : D(u,v) \leq P(x) \]
		Ein $(x^*, u^*, v^*)$ ein SP von $\mathcal{L} \Leftrightarrow P(x^*) = D(u^*,v^*)$
	\end{field}
	\xfield{ Wann sind die \textit{primale} und die \textit{duale} Aufgaben beide L"osbar? Was sind dann die L"osungen?}
	\xfield{ Falls es ein Sattelpunkt $(x^*,u^*,v^*)$ gibt. Die L"osungen sind dann $P(x^*)$ und $D(u^*,v^*)$.}
	\xfield{ Wie hei"st $P(x^*) - D(u^*,v^*)$? }
	\xfield{\textit{Dualit"atsl"ucke}}	
\end{note}

\begin{note}
	\xplain{46fb1e9f-3833-4347-b4f8-f6d6f7e27d31}
	\tags{2.Ordnung}
	\begin{field}
		Sei $(x^*, u^*, v^*)$  ein KKT-Punkt. Dann gilt es $d\tran \nabla f(x^*) \geq 0$.
		 Was sagt uns es, wenn $>$ oder $=$ gelten?
	\end{field}
	\xfield{Bei "$>$" \ ist $d$ eine Anstiegsrichtung von $f$. "$=$" gibt uns keine Aussage dar"uber.}
	\xfield{Wie lautet die \textit{notwendige optimalit"ats Bedingung 2.Ordnung} f"ur eine Lokale L"osung? Was wird vorausgesetzt?}
	\begin{field}
		Wenn $x*$ eine lokale L"osung und \textbf{LICQ} gilt, dann:
		\[d\tran \nabla_{xx} \mathcal{L}(x^*,u^*,v^*)d \geq 0 \quad \forall d\in L^+(x^*,u^*)\]
	\end{field}
	\xfield{Wie ist $L^+(x, u)$ definiert?}
	\begin{field}
		\[L^+(x,u) \defeq \set{d \in L(x) \ \big| \  \nabla g_i(x)\tran d = 0 \text{ falls } u_i > 0} \subseteq L(x)\]
	\end{field}
	\xfield{Wann gilt $d\in L(x^*) \wedge d\tran \nabla f(x^*)=0 \Rightarrow d \in L^(x^*, u^*)$?}
	\xfield{ Falls \textbf{MFCQ} in $x^*$ erf"ullt ist.}
	\xfield{ Wie lautet die \textit{hinreichende Optimalit"ats Bedingung 2.Ordnung}? Was wird genau vorausgesetzt?}
	\begin{field}
		$f, g, h$ st"atig zweimal diff'bar. $(x^*,u^*,v^*)$ ein KKT-Punkt. $x^*$ ist eine \textit{strenge} lokale L"osung der OA, falls: 
		\[d \tran \nabla_{xx} \mathcal{L}(x^*,u^*,v^*) d > 0 \quad \forall d \in L^+(x^*, u^*) / \set{0} \]
	\end{field}
\end{note}

\begin{note}
	\xplain{a6fb041f-4cd6-4dd6-8d2e-e2c22ab0dcc2}
	\tags{}
	\xfield{Wie definiert man eine Abstiegsrichtung?}
	\begin{field}
		$d$ ist eine Abstiegsrichtung eine Funktion $f$ an der Stelle $x$, falls $\exists \bar{\alpha} > 0$, so dass
		\[f(x + \alpha d ) < f(x) \quad \forall \alpha \in (0, \bar{\alpha} ] \]
	\end{field}
	\xfield{ Wann ist ein Vektor $d$ eine Abstiegsrichtung einer Funktion $f$ an der Stelle $x$?}
	\xfield{ Falls $\nabla f(x) \tran d < 0$}
\end{note}

\begin{note}
	\xplain{bf7910cf-fdca-46d4-9667-432bb3eb6b9b}
	\tags{}
	\xfield{Wie l"auft das Gradientenverfahren mit Cauchy -- Schrittweitenwahl ab?}
	\begin{field}
		\begin{description}
		\item[S1 Initialisierung] W"ahle $x^0 \in \Rn, a >0$ setze $k=0$.
		\item[S2 Abbruchtest] Stoppe falls $\nabla f(x^k) = 0$.
		\item[S3 Abstiegsrichtung] Setze $d^k = - \nabla f(x^k)$.
		\item[S4 Schrittweite] Berechne $\alpha_k \geq 0$, so dass
		      \[f(x^k + \alpha_k d^k) \leq  f(x^k + \alpha d^k) \forall \alpha \in [0, a].\]
		
		\item[S5 Update] Setze $x^{k+1} = x^k + \alpha_k d$ und $k = k +1$. Gehe zu S2.
		\end{description}
	\end{field}
	% TODO: Gradienten"hnliche Richtungen.
\end{note}

\begin{note}
	\xplain{e61d7fa2-68e9-4ac0-ad8c-4bfcc337a843}
	\tags{}
	\xfield{Beschriebe das globalisierte Newton-Verfahren (mit Amijo -- Schrittweite). }
	\begin{field}
		\small
		\begin{description}
			\item[S1 Initialisierung] W"ahle $x^0 \in \Rn, \delta \in (0,1), \rho > 0, q >0$, setze $k=0$.
			\item[S2 Abbruchtest] Stopp falls $\nabla f(x^k) =0$.
			\item[S3 Abstiegsrichtung] $H_k = \nabla^2 f(x^k)$, berechne $d^k$ aus  
			\[ H_k d = - \nabla f(x^k).\]
			    Falls nicht m"oglich, oder $\nabla f(x^k) \tran d^k \leq - \rho \norm{d^k}^q$ nicht erf"ullt, dann
			    setze $d^k = -\nabla f(x^k)$.
			\item[S4 Schrittweite] \[\alpha_k = \max\set{\alpha \in S \big | f(x^k + \alpha d^k) \leq f(x^k) + \delta \alpha \nabla f(x^k) \tran d^k }\]
			\item[S5 Update] Setze $x^k = x^k + \alpha_k d^k$ und $k=k+1$ Gehe zu S2.
		\end{description}
	\end{field}
	\xfield{ Wann hat die Folge $\{x^k\}$, erzeugt vom globalisierten Newton Verfahren, mindestens einen H"aufungspunkt?}
	\xfield{ Falls die Niveaumenge $W(x^0) \defeq \set{x \in \Rn \big | f(x) \leq f(x^0)}$ beschr"ankt ist. }
	\xfield{ Wann ist die Niveaumenge $W(x^0)$ kompakt (also auch beschr"ankt)?}
	\xfield{Es reicht die gleichm"a"sige Konvexit"at der Zielfunktion aus.}	
\end{note}

\begin{note}
	\xplain{925c20a8-ea43-4e27-86c6-a88089a616da}
	\tags{}
	\xfield{ Definiere eine \textbf{Q -- lineare Konvergenz}.}
	\begin{field}
		Seien $\{z^k\} \subset \R{l}, z^* \in \R{l}, z^* \not\in \{z^k\}$.
	    Die Folge konvergiert \textbf{Q--linear} gegen $z^*$, falls $\exists \sigma \in (0,1), k_0 \in \N$, so dass $\forall k \geq k_0$
	    \[\frac{\norm{z^{k+1} - z^*}}{\norm{z^k - z^*}} \leq \sigma . \]
	\end{field}
		\xfield{ Definiere eine \textbf{Q -- superlineare Konvergenz}.}
	\begin{field}
		Seien $\{z^k\} \subset \R{l}, z^* \in \R{l}, z^* \not\in \{z^k\}$.
	    Die Folge konvergiert \textbf{Q--superlinear} gegen $z^*$, falls
	    \[ \lim_{k\rightarrow \infty} \frac{\norm{z^{k+1} - z^*}}{\norm{z^k - z^*}} = 0 . \]
	\end{field}
		\xfield{ Definiere eine \textbf{Q -- Ordnung $\tau$ Konvergenz}.}
	\begin{field}
		Seien $\{z^k\} \subset \R{l}, z^* \in \R{l}, z^* \not\in \{z^k\}$.
	    Die Folge konvergiert mit der \textbf{Q--Ordnung $\mathbb{\tau}$ } gegen $z^*$, falls $\lim_{k \rightarrow \infty} z^k = z^*$
	     und $\exists\sigma >0 , \tau > 1$, so dass $\forall k \in \N$:
	    \[\limsup_{k \rightarrow \infty }\frac{\norm{z^{k+1} - z^*}}{\norm{z^k - z^*}^\tau} \leq \sigma . \]
	\end{field}
	\begin{field}
		Was bedeutet, dass eine Folge $\{z^k\}$ gegen $z^*$ \textbf{R--linear}, \textbf{R--superlinear}, oder mit \textbf{R--Ordnung $\tau$} konvergiert?
	\end{field}
	\begin{field}
		$\exists$ eine Folge ${\mu_k} \subset (0, \infty) $, so dass
		\[\norm{z^k - z^*} \leq \mu_k \quad \forall k \in \N\]
		und $\{ \mu_k\}$ mit der entsprechenden Q--Konvergenz gegen $0$ konvergiert. 
	\end{field}
\end{note}

\begin{note}
	\xplain{273d8d2b-6081-45ae-95e7-d772482cd732}
	\tags{}
	\xfield{Wie ist die \textit{quadratische Modellfunktion} bei dem Trust-Region-Verfahren  definiert?}
	\xfield{\[m_k(p) \defeq f(x^k) + \nabla f(x^k) \tran p + \frac{1}{2} p\tran B_k p ,\] mit $B_k \in \R{n \times n}$ symmetrisch, $p=x - x^k$. }
	\xfield{Wie ist der \textbf{G"utema"s} (Trust-Region-Verfahren) definiert?}
	\xfield{\[\rho_k\ \defeq \frac{f(x^k) - f(x^k + p )}{m_k(0) - m_k(p) }\]}
	\xfield{Wie sieht der Trust-Region-Verfahren Algorithmus aus?}
	\begin{field}
		\small
		\begin{description}
			\item[S1] W"ahle $x^0 \in \Rn, \Delta_0 > 0, \ \ 0< \eta_1 < \eta_2 < 1, \ \ {0 < \sigma_1 < 1 < \sigma_2}$. S"atze $k=0$.
			\item[S2] Stoppe falls $\nabla f(x^k) = 0$.
			\item[S3] W"ahle symmetrische Matrix $B_k $ Bestimme die L"osung von 
			\[m_k(p) \rightarrow \min \quad , \norm{p} \leq \Delta_k\]
			\item[S4] Berechne $\rho_k$. \\
				Falls $\rho_k \leq \eta_1$ setze $\Delta_{k+1} = \sigma_1 \Delta_k$ \\
				Falls $\rho_k \in (\eta_1, \eta_2)$ setze $\Delta_{k+1} =  \Delta_k$ \\
				Falls $\rho_k \geq \eta_2$ setze $\Delta_{k+1} = \sigma_2 \Delta_k$ \\
			\item[S5] Falls $\rho_k > \eta_1$, setze $x^{k+1} = x^k + p^k, k = k+1$. Gehe zu S2.
			\item[S6] Setze $x^{k+1} = x^k, k=k+1$ gehe zu S3.
		\end{description}
	\end{field}
\end{note}
\end{document}
