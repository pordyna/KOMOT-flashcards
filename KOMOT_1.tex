% !TeX spellcheck = de_DE
% !TEX encoding = UTF-8 Unicode
%%%%%%%%%%%%%%%%%%%%%%%%%%%%%%%%%%%%%%%%%%%%%%%%%%%%
% The first part of the header needs to be copied
%       into the note options in Anki.
%%%%%%%%%%%%%%%%%%%%%%%%%%%%%%%%%%%%%%%%%%%%%%%%%%%%

% layout in Anki:
\documentclass[9pt]{article}
\usepackage[a4paper]{geometry}
\geometry{paperwidth=.5\paperwidth,paperheight=100in,left=2em,right=2em,bottom=1em,top=2em}
\pagestyle{empty}
\setlength{\parindent}{0in}
 
% hyphenation:
\usepackage[ngerman]{babel}

% encoding:
\usepackage[T1]{fontenc}
\usepackage[utf8]{inputenc}
\usepackage{lmodern}

% packages:
\usepackage{parskip}
\usepackage[free-standing-units=true]{siunitx}
% Writng si units, numbers, list of numbers etc.
\usepackage{gensymb}
% Unified typseting of units outside of siuitx
\usepackage{amsmath}
\usepackage{amsfonts}  
% Math features
\usepackage{esdiff}
% Typeseting of (partial)derivatives 
\usepackage{commath}
% More  derivatives. Not so nice like esdiif but adds
% \dif comand for upright d in math mode.
\usepackage{bm}
% Bold font in math mode
\usepackage{esint}
% Some fancy integrals signs. Mutilpe integrals
\usepackage{enumerate}
% Different styles for enumerate lists
\usepackage{multirow}
% More advanced tabular
\usepackage{physics}
% Lots of usefull comannds for physicists. Vektors, nablas etc.
\usepackage{amssymb, xfrac,bbold}
% extra fonts and symbols 
\usepackage{mathtools}
% extension to amsmath, fixes, meany new tool
\usepackage{isotope}
\usepackage{empheq}
%commands:
\usepackage{ifthen}

\newcommand*{\tran}{^{\mkern-1.5mu\mathsf{T}}}
\newcommand{\Rn}{\mathbb{R}^n}
\newcommand{\Rk}{\mathbb{R}^k}
\newcommand{\defeq}{\coloneqq}
\newcommand{\R}[1]{%
\ifthenelse{\equal{#1}{}}%
	{\mathbb{R}}
	{\mathbb{R}^{#1}}}
\newcommand{\C}[1]{%
	\ifthenelse{\equal{#1}{}}
	{\mathbb{C}}
	{\mathbb{C}^{#1}}}%	
\renewcommand{\vec}[1]{\underline{#1}}
\DeclarePairedDelimiter{\innerprod}\langle\rangle
%\DeclarePairedDelimiter{\norm}{\lVert}{\rVert} 
\newcommand{\Fr}{\mathcal{F}}
\newcommand{\Hi}{\mathcal{H}}
%%%%%%%%%%%%%%%%%%%%%%%%%%%%%%%%%%%%%%%%%%%%%%%%%%%%
% Following part of header NOT to be copied into
%            the note options in Anki.
%          ! Anki will throw an errow !
%%%%%%%%%%%%%%%%%%%%%%%%%%%%%%%%%%%%%%%%%%%%%%%%%%%%%
%
%  pdf layout:
%
\geometry{paperheight=74.25mm}
\usepackage{pgfpages}
\pagestyle{empty}
\pgfpagesuselayout{8 on 1}[a4paper,border shrink=0cm]
\makeatletter
\@tempcnta=1\relax
\loop\ifnum\@tempcnta<9\relax
\pgf@pset{\the\@tempcnta}{bordercode}{\pgfusepath{stroke}}
\advance\@tempcnta by 1\relax
\repeat
\makeatother
% 
%  notes, fields, tags:
%
\newcommand{\xfield}[1]{
        #1\par
        \vfill
        {\tiny\texttt{\parbox[t]{\textwidth}{\localtag\globaltag\hfill\uuid}}}
        \newpage}
\newenvironment{field}{}{\vfill
        {\tiny\texttt{\parbox[t]{\textwidth}{\localtag\globaltag\hfill\uuid}}} \newpage}
\newif\ifnote
\newenvironment{note}{\notetrue}{\notefalse}
\newcommand{\localtag}{}
\newcommand{\globaltag}{}
\newcommand{\uuid}{}
\newcommand{\tags}[1]{
    \ifnote 
        \renewcommand{\localtag}{#1}
    \else
        \renewcommand{\globaltag}{#1}
    \fi 
    }
\newcommand{\xplain}[1]{\renewcommand{\uuid}{#1}}
%
%%%%%%%%%%%%%%%%%%%%%%%%%%%%%%%%%%%%%%%%%%%%%%%%%%%%
% The following line again needs to be copied 
% into Anki:
\begin{document}
%%%%%%%%%%%%%%%%%%%%%%%%%%%%%%%%%%%%%%%%%%%%%%%%%%%%

\tags{KOMOT::Optimierungsprobleme}

\begin{note}
  \xplain{UUID}
  % 1.1.2 
  \tags{aufgabenstellung}
  \xfield{Was versteht man unter einer \textit{Zielfunktion}?}
  \begin{field}   
  	Die Funtktion $f: G \rightarrow \R{}$, die minimiert wird. 
  \end{field}
  \xfield{Was ist der \textit{zul"assiger Bereich} $G$ ?}
  \begin{field}
  	Definitionsbereich der Zielfunktion. $G \subseteq \Rn$ .
  \end{field}
  \xfield{Was verstehen wir unter einer (globalen) L"osung einer OA Aufgabe.}
  \xfield{Ein $x^* \in G$ das die Zielfunktion minimiert. } 
  \xfield{Was ist eine \textit{lokale L"osung} $x^*$ einer OA Aufgabe?}
  \begin{field}
  	\begin{equation*}
  		f(x^*) \leq f(x) \quad  \forall x \in G \cap U(x^*) ,
  	\end{equation*}
  	 und es existiert so eine Umgebung von $x^*$.
  \end{field}
  \xfield{Was ist eine \textit{isolierte L"osung}?}
  \begin{field}
  	Es existiert eine Umgebung $U(x*)$, so dass $f(x^*) < f(x)$.
  	Bzw. es gibt keine witeren lokalen Lo"sungen in der Umgebung. 
  \end{field}
  \xfield{ Wie hei"st $f_\text{min} \defeq f(x^*)$ ?}
  \xfield{\textit{Optimalwert} oder \textit{Minimalwert}}
\end{note}

\begin{note}
	\xplain{UUID}
	% 1.1.3
	\tags{konvexitaet}
	\xfield{Eine Menge $G\subseteq\Rn$ hei"st \textit{konvex}, wenn ...}
	\begin{field}
		$\forall_{x,y \in G}$ die Verebindungstrecke zwischen den Punkten auch in $G$ liegt. Formel: 
		\[\lambda x + (1 - \lambda )y \in G, \quad \forall (x,y,\lambda ) \in (G \times G \times (0, 1)) \]
	\end{field}
	\xfield{Sei $G$ \textit{konvex}. Eine Funktion $f:G \rightarrow \R{}$ hei"st \textit{konvex auf} $G$, wenn... }
	\begin{field}
	$\forall (x,y,\lambda ) \in (G \times G \times (0, 1))$:
		\[f(\lambda x + (1-\lambda) y \leq \lambda f(x) + (1 - \lambda) f(y).\]
	\end{field}
	\xfield{Wann ist eine Funktion \textit{streng konvex} auf einer kompakten Menge $G$? }
	\xfield{Wie bei normalen konvexit"at, aber mit $<$ statt $\leq$.}
		\xfield{Sei $G$ \textit{konvex}. Eine Funktion $f:G \rightarrow \R{}$ hei"st \textit{gleichm"a"sig konvex auf} $G$, wenn... }
	\begin{field}
	$\exists \gamma > 0$, so dass $\forall (x,y,\lambda ) \in (G \times G \times (0, 1))$:
		\[f(\lambda x + (1-\lambda) y \leq \lambda f(x) + (1 - \lambda) f(y) - \gamma \lambda (1 - \lambda ) \norm{x-y}^2 \]
	\end{field}
	\end{note}
	
\begin{note}
	\xplain{UUID}
	% 1.1 Lemma
	\begin{field}
		$B$ offen $G$ konvex und $G\subseteq B \subseteq \Rn$, $f: B \rightarrow \R{}$ diff'bar.
		$f$ \textit{konvex} g.d.w. ... 
	\end{field}
	\begin{field}
	    $\forall_{x,y \in G}$: 
		\[f(y) - f(x) \geq \nabla f(x)\tran (y-x). \]
	\end{field}
	\begin{field}
		$B$ offen $G$ konvex und $G\subseteq B \subseteq \Rn$, $f: B \rightarrow \R{}$ diff'bar.
		$f$ \textit{streng konvex} g.d.w. ... 
	\end{field}
	\begin{field}
	    $\forall_{x,y \in G}, x \neq y$: 
		\[f(y) - f(x) > \nabla f(x)\tran (y-x). \]
	\end{field}
	\begin{field}
		$B$ offen $G$ konvex und $G\subseteq B \subseteq \Rn$, $f: B \rightarrow \R{}$ diff'bar.
		$f$ \textit{gleichm"a"sig konvex} g.d.w. ... 
	\end{field}
	\begin{field}
	    $\exists \gamma >0$, so dass $\forall_{x,y\in G}$: 
		\[f(y) - f(x) \geq \nabla f(x)\tran (y-x) + \gamma\norm{x - y}^2. \]
	\end{field}
\end{note}

\begin{note}
	\xplain{UUID}
	% 1.2
	\tags{definitheit}
	\begin{field}
		Wann ist eine quadratische Matrix  $M \in \R{n \times n }$\textit{positiv semidefinit}?
	\end{field}
	\begin{field}
		wenn $s\tran M s \geq 0 \forall s \in \Rn$. ($\Leftrightarrow$: Alle Eigenwerte $ \geq 0$.)
	\end{field}
	\xfield{Wann ist eine quadratische Matrix  $M \in \R{n \times n }$\textit{positiv definit}?}
	\begin{field}
		wenn $s\tran M s > 0, \quad \forall s \in \Rn / \set{0} $. 
		($\Leftrightarrow$: Alle Eigenwerte $ > 0$.)
	\end{field}
\end{note}

\begin{note}
	\xplain{UUID}
	\tags{}
	% Lemma 1.2
	\begin{field}
		$G \subseteq \Rn$ offen und konvex, $f: G \rightarrow \R{}$ zweimal stetig diff'bar.
		 Dann $f$ is \textit{konvex} auf $G$, genau dann wenn... 
	\end{field}
	\begin{field}
		$\forall x \in G: \nabla^2f(x)$ positiv semidefinit.
	\end{field}
	\begin{field}
		$G \subseteq \Rn$ offen und konvex, $f: G \rightarrow \R{}$ zweimal stetig diff'bar.
		 Dann $f$ is \textit{streng konvex} auf $G$, genau dann wenn... 
	\end{field}
	\begin{field}
		$\forall x \in G: \nabla^2f(x)$ positiv definit.
	\end{field}
	\begin{field}
		$G \subseteq \Rn$ offen und konvex, $f: G \rightarrow \R{}$ zweimal stetig diff'bar.
		 Dann $f$ is \textit{streng konvex} auf $G$, genau dann wenn... 
	\end{field}
	\begin{field}
		$\exists \gamma > 0$, so dass $\forall s,x \in G$:
		\[s\tran \nabla^2f(x)s \geq \gamma \norm{s}^2\] 
	\end{field}
\end{note}

\begin{note}
	\xplain{UUID}
	\tags{}
	%Theorem 1.1
	\begin{field}
		Die Zielfunktion sei konvex, was gibt uns das( bezogen auf L"osugnen)?
	\end{field}
	\begin{field}
    		Jede lokale L"osung ist auch eine globale L"osung
	\end{field}
	\begin{field}
		Die Zielfunktion sei streng konvex, was gibt uns das( bezogen auf L"osugnen)?
	\end{field}
	\begin{field}
		Es gibt h"ochstens eine globale L"osung.
	\end{field}
	\begin{field}
		Die Zielfunktion sei gleichm"a"sig konvex, was gibt uns das ( bezogen auf L"osugnen)?
	\end{field}
	\begin{field}
		Falls $G$ nicht nur konvex aber auch abgeschlossen und nichtleer, dann besitzt die OA \emph{genau eine} L"osung. 
	\end{field}
\end{note}

\begin{note}
	\xplain{UUID}
	\tags{}
	\begin{field}
		Was ist die Definition der \emph{quasikonvexit"at}?
	\end{field}
	\begin{field}
		$G \subseteq \Rn$ konvex. $f:G \rightarrow \R{}$ hei"st \textit{quasikonvex} auf $G$, wenn 
		\[f( \lambda x + (1- \lambda ) y) \leq \max\set{f(x), f(y) }\]
	\end{field}
	\begin{field}
	Was bedeutet, dass eine Funktion \textit{pseudokonvex} ist?
	\end{field}
	\begin{field}
		Sei $G$ konvex, $B$ offen mit $G \subseteq B \subseteq \Rn$. Sei $f: B \rightarrow \R{}$ diff'bar.
		$f$ ist pseudokonvex auf G, wenn $\forall x,y \in G$:
		\[(y-x)\tran\nabla f(x) \geq 0 \Rightarrow f(y) \geq f(x).\]
	\end{field}
	\xfield{Was ist st"arker, \textit{pseudokonvexit"at} oder \textit{quasikonvexit"at}?}
	\xfield{\textit{pseudokonvexit"at}}
\end{note}

%1.2
\tags{KOMOT::Optimalitaetsbedinugngen}
%1.2.1
\begin{note}
	\xplain{UUID}
	\tags{}
	\xfield{Definiere den Kegel der zul"assigen Richtungen in $x\in G$.}
	\begin{field}
		\[Z(x) \defeq \text{cone}\set{d \in \Rn \ | \ x + \alpha d \in G, \ \ 
		\forall \alpha \in \left[0, 1 \right] } \]
		, wobei $\text{cone}(S) \defeq \set{\lambda s \ | \ s\in S, \ \ \lambda \in\left[0, \infty \right)}$.
	\end{field}
	\begin{field}
		Sei $G$ konvex, die Zielfunktion $f$ [... (1)], $x^*$ [... (2)], und es gilt [... (3)], dann ist $x^*$ eine globale L"osung der OA.
	\end{field}
	\begin{field}
		\begin{enumerate}
			\item pseudokonvex
			\item  eine lokale L"osung
			\item $\nabla f(x^*)\tran(x-x^*) \geq 0, \ \ \forall x \in G $
		\end{enumerate}
	\end{field}
\end{note}

\end{document}
